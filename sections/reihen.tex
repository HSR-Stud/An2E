\section{Reihen\formelbuch{420, 427, 1045}}

\subsection{Zahlenreihen\formelbuch{422}}
$ s_n = \sum\limits_{k=1}^{n} a_k \qquad $ ist eine (unendliche) Reihe. Sie ist die Folge von Partialsummen einer bestehenden Folge $a_n$.

\subsubsection{Konvergenz, Divergenz\formelbuch{422}}
Konvergiert die Reihe $< s_n >$ gegen die Summe $ s = \sum\limits_{k=1}^{\infty} a_k $ so ist sie konvergent. 
Existiert der GW nicht, so ist sie divergent.

\subsubsection{Konvergenzkriterien}

\paragraph{Cauchy-Kriterium} 
Wenn zu jedem $\varepsilon > 0$ ein Index $n_0$ existiert, so dass für alle $m > n > n_0$ gilt: \\
$\left| \sum\limits_{k=n}^m a_k \right| < \varepsilon$, dann konvergiert die Reihe, ansonsten divergiert sie.

\paragraph{lim = 0\formelbuch{423}}
Wenn die Reihe $ \sum\limits_{n=1}^{\infty} a_n $ konvergent ist, so ist $\lim\limits_{n \to \infty} a_n = 0$. \hspace{2cm} Aber NICHT UMGEKEHRT!

\paragraph{Divergenz}
Ist $<a_n>$ divergent oder ist $\lim\limits_{n \to \infty} a_n \neq 0$, so ist die Reihe $ \sum\limits_{n=1}^{\infty} a_n $ divergent.

\paragraph{Majorantenkriterium\formelbuch{429}}
Ist die Reihe $ \sum\limits_{n=1}^{\infty} c_n $ konvergent, so konvergiert auch die Reihe $ \sum\limits_{n=1}^{\infty} a_n $ für $|a_n| \leq c_n$ (absolut). \\
Dies gilt auch für $|a_n| \leq c_n$ erst ab einer Stelle $n_0 \in \mathbb{N}$.

\paragraph{Minorantenkriterium}
Ist die Reihe $ \sum\limits_{n=1}^{\infty} d_n $ gegen $+\infty$ divergent, so gilt dies auch für die Reihe $ \sum\limits_{n=1}^{\infty} a_n $ 
bei $a_n \geq d_n$. \\ Dies gilt auch für $a_n \geq d_n$ erst ab einer Stelle $n_0 \in \mathbb{N}$.

\paragraph{Reziprokkriterium}
$ s = \sum\limits_{n=1}^{\infty} \frac{1}{n^\alpha} $ ist konvergent für $\alpha > 1$ und divergent für $\alpha \leq 1$.

\paragraph{Quotientenkriterium\formelbuch{424}}
$ \lim\limits_{n \to \infty} \left|\frac{a_{n+1}}{a_n}\right| = \alpha $ der Reihe $ \sum\limits_{n=1}^{\infty} a_n $ \\
$\alpha < 1 \Rightarrow$ (absolut) konvergent \hspace{3cm}
$\alpha > 1 \Rightarrow$ divergent \hspace{4cm} 
$\alpha = 1 \Rightarrow$ keine Aussage!

\paragraph{Wurzelkriterium\formelbuch{424f}}
$ \lim\limits_{n \to \infty} \sqrt[n]{\left|a_n\right|} = \alpha $ der Reihe $ \sum\limits_{n=1}^{\infty} a_n $ \\
$\alpha < 1 \Rightarrow$ (absolut) konvergent\hspace{3cm}
$\alpha > 1 \Rightarrow$ divergent \hspace{4cm} 
$\alpha = 1 \Rightarrow$ keine Aussage!

\paragraph{Integralkriterium\formelbuch{425}}
$ \sum\limits_{n=1}^{\infty} f(n) $ ist konvergent, wenn das uneigentliche Integral $ \int\limits_{1}^{\infty} f(x) dx $ konvergent ist. \\
Gilt nur, wenn $f$ auf $ [1, \infty) $ definiert und monoton fallend ist. Zudem muss $ f(x) \geq 0 $ für alle $x \in [1, \infty)$ sein.
 
\paragraph{Leibniz-Kriterium\formelbuch{426}}
Die \textbf{alternierende} Reihe $ \sum\limits_{n=1}^{\infty} a_n $ ist konvergent, wenn die Folge $<\left|a_n\right|>$ eine monoton fallende Nullfolge ($\lim\limits_{n \to \infty}
\left|a_n\right| = 0 $) ist.
\\ Monotonie mittels Verhältnis ($ \left|\frac{a_{n+1}}{a_n}\right| $), Differenz ($ |a_{n+1}| - |a_n| $) oder \textit{vollständiger Induktion} beweisen.\\ 

\paragraph{Abschätzung Restglied einer alternierenden konvergenten Reihe\formelbuch{426,430}}\qquad $|R_n| = |s-s_n|\leq |a_{n+1}|$


\subsubsection{Bedingte und Absolute Konvergenz\formelbuch{425}}
Eine Reihe $\sum\limits_{n=1}^{\infty}a_n$ heisst \textbf{absolut konvergent}, wenn die
Reihe $\sum\limits_{n=1}^{\infty}|a_n|$ konvergent ist.\\
\textbf{Bedingt Konvergent:} Eine Reihe hat durch Umordnen einen anderen
Grenzwert oder wird divergent.\\
\textbf{Unbedingt Konvergent:} Durch Umordnen ändert sich der Grenzwert nicht.

\subsubsection{Produkt von absolut konvergenten Reihen\formelbuch{426}} 
Gegeben sei: $\sum a_n=a$, \quad $\sum b_n=b, \quad \sum c_n = (\sum a_n) \cdot (\sum b_n) = c \quad $ so ist
$ \quad c_n=\sum a_kb_{n-k+1} \quad $ und $ \quad c = a \cdot b $



\subsection{Potenzreihen}

%\paragraph{Definition\formelbuch{432}} 

Die Reihe $ \sum\limits_{n=0}^{\infty} a_n (x-x_0)^n $ heisst Potenzreihe mit Entwicklungspunkt $x_0$ und Koeffizienten $a_n$.

\begin{tabular}{lll}
\textbf{Geometrische Reihe\formelbuch{19}}
  & $ \frac{a}{1-x} = a \cdot \sum\limits_{n=0}^{\infty} x^n$
  & $(|x| < 1) \qquad$ Beidseitiges $\int \quad\Rightarrow\quad -a \cdot \ln{|x-1|} 
= a \cdot \sum\limits_{n=1}^{\infty} \frac{x^{n}}{n} $ \\
\textbf{Binominalreihe} 
  & $ (1+x)^\alpha = \sum\limits_{n=0}^\infty \binom{\alpha}{n} x^n$
  & $x \in (-1,1)$ \\
\textbf{Taylor-Reihe\formelbuch{434, 1045}}
  & $ \sum\limits_{n=0}^{\infty} \frac{f^{(n)}(x_0)}{n!}\cdot(x-x_0)^n$
  & Taylor-Reihe von f bezüglich der Stelle $x_0$ \\
\textbf{E-Funktion}
  & \multicolumn{2}{l}{$e = \lim\limits_{n\to\infty} \left(1+\frac{1}{n}\right)^n = 
  \sum\limits_{k=0}^{\infty}{\frac{1}{k!}} = 1 + \frac{1}{1} + \frac{1}{1\cdot 2} +
  \frac{1}{1\cdot 2\cdot 3}  + \frac{1}{1\cdot 2\cdot 3\cdot4} + \cdots$}
\end{tabular}

\subsubsection{Konvergenz\formelbuch{432}}
Gegeben sei die Potenzreihe $ \sum\limits_{n=0}^{\infty} a_n x^n $ mit $ \lim\limits_{n \to \infty} \sqrt[n]{|a_n|} = a $ \\
Für $ a=0 $ ist die Potenzreihe für alle $ x \in \mathbb{R} $ absolut konvergent. \\
Für $ a>0 $ ist die Potenzreihe für alle $x$ mit 
$ \left\{   
    \begin{array}{l} 
      |x| < \frac{1}{a} = \rho \Rightarrow \text{ absolut konvergent.} \\
      |x| > \frac{1}{a} = \rho \Rightarrow \text{ divergent.}
    \end{array} 
  \right. $ \\
Ist die Folge $<\sqrt[n]{|a_n|}>$ nicht beschränkt, so ist die Potenzreihe nur für $x=0$ konvergent.

\subsubsection{Konvergenzradius\formelbuch{432}}
Jeder Potenzreihe kann ein Konvergenzradius $\rho$ zugeordnet werden. Wobei gilt $\rho = \frac{1}{a}$ mit $a = \lim\limits_{n \to \infty} \sqrt[n]{|a_n|} $. \\
Für $a = 0$ gilt $\rho = \infty$. Wenn a nicht exisitiert (Folge divergent) ist $\rho = 0$. \\
Berechnung mittels Quotientenkriterium: $ \rho = \lim\limits_{n \to \infty} \left| \frac{a_n}{a_{n+1}} \right|$

\subsubsection{Differentiation}
Alle Potenzreihen mit einem $\rho > 0$ sind für alle $x \in (-\rho, \rho)$ beliebig oft (gliedweise) differenzierbar. \\
Der Potenzradius $\rho$ ist bei allen Ableitungen gleich demjenigen der Ursprungsfunktion. $\rho_{f} = \rho_{f^{(i)}}$.
$$ f(x) = \sum\limits_{n=0}^{\infty} a_n x^n  \qquad 
   f'(x) = \sum\limits_{n=1}^{\infty} n \cdot a_n x^{n-1 } \qquad 
   f''(x) = \sum\limits_{n=2}^{\infty} n(n-1) \cdot a_n x^{n-2} \qquad 
   f^{(i)}(x) = \sum\limits_{n=i}^{\infty} n(n-1)\cdot \ldots \cdot (n-i+1)\cdot a_n x^{n-i} $$ 
\textbf{Bemerkung:} Startwert ($n=0$) nur erhöhen, wenn bei $x^n, n$ negativ werden würde!

\subsubsection{Integration}
\paragraph{Unbestimmtes Integral}
$\int \sum\limits_{n=0}^{\infty} a_n x^n dx = 
\sum\limits_{n=0}^{\infty} a_n \int x^n dx = 
\sum\limits_{n=0}^{\infty} \frac{a_n}{n+1}\cdot x^{n+1} \qquad \text{ für alle } x \in (-\rho, \rho).$
\paragraph{Bestimmtes Integral}
$\int\limits_0^x \sum\limits_{n=0}^{\infty} a_n t^n dt = 
\sum\limits_{n=0}^{\infty} \frac{a_n}{n+1}\cdot x^{n+1} \qquad \text{ für alle } x \in (-\rho, \rho).$
